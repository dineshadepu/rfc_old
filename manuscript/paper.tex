\documentclass[preprint,12pt]{elsarticle}

\usepackage{caption}
\usepackage{hyperref}
\usepackage{graphicx}
\usepackage{subcaption}
\usepackage{amssymb}
\usepackage{amsmath}
\usepackage{multirow}
\usepackage[utf8]{inputenc}
\usepackage{cleveref}

% For the TODOs
\usepackage{xcolor}
\usepackage{xargs}
\usepackage[colorinlistoftodos,textsize=footnotesize]{todonotes}
\newcommand{\todoin}{\todo[inline]}
% from here: https://tex.stackexchange.com/questions/9796/how-to-add-todo-notes
\newcommandx{\unsure}[2][1=]{\todo[linecolor=red,backgroundcolor=red!25,bordercolor=red,#1]{#2}}
\newcommandx{\change}[2][1=]{\todo[linecolor=blue,backgroundcolor=blue!25,bordercolor=blue,#1]{#2}}
\newcommandx{\info}[2][1=]{\todo[linecolor=OliveGreen,backgroundcolor=OliveGreen!25,bordercolor=OliveGreen,#1]{#2}}

%Boldtype for greek symbols
\newcommand{\teng}[1]{\ensuremath{\boldsymbol{#1}}}
\newcommand{\ten}[1]{\ensuremath{\mathbf{#1}}}

\usepackage{lineno}

\journal{}

\begin{document}

\begin{frontmatter}

  \title{Rigid fluid coupling}
  \author[IITB]{Adepu Dinesh \corref{cor1}}
  \ead{adepu.dinesh.a@gmail.com} \author[IITB]{Prabhu Ramachandran}
  \ead{prabhu@aero.iitb.ac.in} \address[IITB]{Department of Aerospace
    Engineering, Indian Institute of Technology Bombay, Powai, Mumbai 400076}

\cortext[cor1]{Corresponding author}

\begin{abstract}
\end{abstract}

\begin{keyword}
%% keywords here, in the form: keyword \sep keyword
{XXX}, {XXX}, {XXX}

%% MSC codes here, in the form: \MSC code \sep code
%% or \MSC[2008] code \sep code (2000 is the default)

\end{keyword}

\end{frontmatter}

% \linenumbers

\section{Introduction}
\label{sec:intro}



\section{Governing equations}
\label{sec:fluid-mechanics}


\section{CTVF formulation}
\label{sec:ctvf-equations}


\section{SPH discretization}
\label{sec:sph-fluid-equations}


\section{Rigid body dynamics}
\label{sec:rbd}


\section{How to handle rigid fluid coupling}
\label{sec:rfc}


\section{Collision between rigid bodies}
\label{sec:contact-force}

\citet{chen2019coupled} has damping model. Also he mentioned the parameters
required in stack of cylinders example.


% \todoin{
% \begin{itemize}
% \item \citet{albano2016modelling} models using pure elastic impingement force
% % Development of the Resolved Fluid-Solid SPH Coupling using Rigid Body Dynamics
% \item \citet{choidevelopment} models using DEM
% \item \citet{zhan2020sph} uses hybrid contact model
% \end{itemize}
% A SPH framework for dynamic interaction between soil and rigid body system
% with hybrid contact method}



\section{Results and discussion}
\label{sec:results}

\subsection{Rigid block of different densities allowed to sink into water}
\label{sec:rigid-block-diff}


\subsection{A 2d and 3d rigid body sliding}
\label{sec:rigid-body-sliding}


\subsection{Bouncing cube on a wall under gravity}
\label{sec:bouncing-cube}


\subsection{Stack of cylinders}
\label{sec:stack-of-cylinders}


\subsection{A test to check the conservation properties of rigid body}
\label{sec:conservation-of-rb-properties}


\subsection{Debris dam break flow}
\label{sec:debris-dam-break-canelas}

\subsubsection{Case I}
\label{sec:debris-dam-break-canelas-case-I}

\subsubsection{Case II}
\label{sec:debris-dam-break-canelas-case-II}


\subsubsection{Case III}
\label{sec:debris-dam-break-canelas-case-III}


\subsection{Dam break with body transport}
\label{sec:dam-break-with-body-transport}

\citet{wang2019numerical}

\subsection{Dam break with multiple bodies transport}
\label{sec:dam-break-with-multiple-bodies-transport}
\citet{wang2019numerical}


\subsection{Cylinders in water collapsed under gravity}
\label{sec:cylinders-collapse-in-water}
\citet{chen2019coupled}


\section{Conclusions}
\label{sec:conclusions}

\section*{References}
\bibliographystyle{model6-num-names}
\bibliography{references}



\end{document}

%%% Local Variables:
%%% mode: latex
%%% TeX-master: "paper"
%%% fill-column: 78
%%% End:
