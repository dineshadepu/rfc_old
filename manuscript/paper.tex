\documentclass[preprint,12pt]{elsarticle}

\usepackage{caption}
\usepackage{hyperref}
\usepackage{graphicx}
\usepackage{subcaption}
\usepackage{amssymb}
\usepackage{amsmath}
\usepackage{multirow}
\usepackage[utf8]{inputenc}
\usepackage{cleveref}

% For the TODOs
\usepackage{xcolor}
\usepackage{xargs}
\usepackage[colorinlistoftodos,textsize=footnotesize]{todonotes}
\newcommand{\todoin}{\todo[inline]}
% from here: https://tex.stackexchange.com/questions/9796/how-to-add-todo-notes
\newcommandx{\unsure}[2][1=]{\todo[linecolor=red,backgroundcolor=red!25,bordercolor=red,#1]{#2}}
\newcommandx{\change}[2][1=]{\todo[linecolor=blue,backgroundcolor=blue!25,bordercolor=blue,#1]{#2}}
\newcommandx{\info}[2][1=]{\todo[linecolor=OliveGreen,backgroundcolor=OliveGreen!25,bordercolor=OliveGreen,#1]{#2}}

%Boldtype for greek symbols
\newcommand{\teng}[1]{\ensuremath{\boldsymbol{#1}}}
\newcommand{\ten}[1]{\ensuremath{\mathbf{#1}}}

\usepackage{lineno}

\journal{}

\begin{document}

\begin{frontmatter}

  \title{Write title here}
  \author[IITB]{Adepu Dinesh \corref{cor1}}
  \ead{adepu.dinesh.a@gmail.com} \author[IITB]{Prabhu Ramachandran}
  \ead{prabhu@aero.iitb.ac.in} \address[IITB]{Department of Aerospace
    Engineering, Indian Institute of Technology Bombay, Powai, Mumbai 400076}

\cortext[cor1]{Corresponding author}

\begin{abstract}
\end{abstract}

\begin{keyword}
%% keywords here, in the form: keyword \sep keyword
{XXX}, {XXX}, {XXX}

%% MSC codes here, in the form: \MSC code \sep code
%% or \MSC[2008] code \sep code (2000 is the default)

\end{keyword}

\end{frontmatter}

% \linenumbers

\section{Introduction}
\label{sec:intro}



\section{Fluid mechanics}
\label{sec:fluid-mechanics}



\section{Rigid body dynamics}
\label{sec:rbd-dynamics}

\subsection{A rigid body on a boundary}

Consider a square rigid body on a surface, the density of the body is 800
kg$m^{-3}$, the simulation is in 2 dimensions. The side of the body is
$0.5$m. The stiffness in the normal direction is $10^5$N$m^{-1}$.

\textbf{The analysis corresponding to the current section is done in file
  code $rigidbodyonasurface.py$}




\section{Rigid body contact model}
\label{sec:rbd-contact-model}

In the SPH literature there are a few ways to model the contact
between rigid bodies. Few ways to note are

\begin{itemize}
\item \citet{albano2016modelling} models using pure elastic impingement force
% Development of the Resolved Fluid-Solid SPH Coupling using Rigid Body Dynamics
\item \citet{choidevelopment} models using DEM
\item \citet{zhan2020sph} uses hybrid contact model
\item \citet{} uses hybrid contact model
\end{itemize}
A  SPH  framework  for  dynamic interaction  between  soil  and  rigid  body  system  with  hybrid contact   metho


\section{Rigid fluid coupling}
\label{sec:rf-coupling}




\section{Results and discussion}
\label{sec:results}
\citet{wu2014two} has experimental solution to cube sinking and cube floating
in a tank.

\subsection{Cylinders collapsing under gravity in a tank}
\label{sec:cylinders-collapsing-dem}

This benchmark is taken from \citet{chen2019coupled} and
\citet{wang2019numerical}. This is to validate the rigid contact model.

Six layers of solid cylinders are stacked, and then suddenly released under
gravity. This is done by 2009 Zhang and Canelas 2016.


\subsection{A rigid box rotating and sinking in viscous liquid}
\label{sec:rigid-box-rotating-sinking}
% Numerical simulation of interactions between free surface and rigid body using
% a robust SPH method, Pengnan Sun 2015

This benchmark is taken from \citet{chen2019coupled} and
\citet{wang2019numerical}. This is to validate the rigid contact model.

Six layers of solid cylinders are stacked, and then suddenly released under
gravity. This is done by 2009 Zhang and Canelas 2016.


\subsection{A block of density 2000 freely falling in to a steady tank}
\label{sec:2000-density-inside-tank}

This benchmark is taken from \citet{qiu20173d}.

A water tank of 150 $mm$ $\times$ 140 $mm$ $\times$ 140 $mm$ and the water
depth in the tank was 131 $mm$. The cube with a side length of 20 $mm$. The
density of the cube is 2120 $kg m^{-3}$. The cube is placed half way before
the release into the fluid. The spring stiffness coefficient of 1 $\times$
$10^{5}$ $N m^{-1}$.


\subsection{A block of density $800.52$ floating in a steady tank}
\label{sec:800-density-floating}

This benchmark is taken from \citet{qiu20173d}, \citet{wang2019numerical}.

A water tank of 150 mm $\times$ 140 mm $\times$ 140 mm and the water depth in
the tank was 52 $mm$. A rectangular wooden block (with a width of 48 mm,
length of 49 mm, height 24mm) whose density is 800.52 kg$m^{-1}$ is
released from the bottom of the tank. We compare the vertical displacement of
the center of mass of the cube.


\subsection{Dam break with body transport}
\label{sec:dam-break-with-body-transport}

This benchmark is taken from \citet{wang2019numerical}.

Here we simulate a solid body transport in a dam break.


\subsection{Dam break with multiple body transport}
\label{sec:dam-break-with-multiple-body-transport}

This benchmark is taken from \citet{wang2019numerical}.

Here we simulate a transport of multiple bodies in a dam break.


\subsection{Cylinders in water collapsed under gravity}
\label{sec:cylinders-in-water-collapsed-under-gravity}

This benchmark is taken from \citet{chen2019coupled}.

Here we simulate a collapse of cylinders submerged in a water under gravity.


\subsection{Single cube transport in 3d dam break with out any obstructions}
\label{sec:single-cube-transport-in-3d-dam-break-with-out-any-obstructions}

This benchmark is taken from \citet{ji2019coupled}.

A single body transport under 3d dam break. Here we don't have any
obstructions. This is done in Canelas 2016 DCDEM paper. This is different from
Amicarelli paper.


\subsection{Three cube transport in 3d dam break with out any obstructions}
\label{sec:three-cube-transport-in-3d-dam-break-with-out-any-obstructions}

This benchmark is taken from \citet{ji2019coupled}.

Three cubes transport under 3d dam break. Here we don't have any obstructions.
This is done in Canelas 2016 DCDEM paper. This is different from Amicarelli
paper.


\section{Conclusions}
\label{sec:conclusions}


\section*{References}
\label{sec:references}

The adaptive fluid solid interaction is done by \citet{hu2019consistent}.


\section*{References}
\bibliographystyle{model6-num-names}
\bibliography{references}



\end{document}

%%% Local Variables:
%%% mode: latex
%%% TeX-master: "paper"
%%% fill-column: 78
%%% End:
